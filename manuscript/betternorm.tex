\documentclass{article}
\usepackage[margin=2.5cm]{geometry}
\usepackage{amsmath}

\title{Computing normalization factors for single-cell RNA-seq data: avoiding problems with zero counts}
\author{Aaron Lun and Karsten Bach}

\begin{document}
\maketitle

\section{Introduction}
Single-cell RNA sequencing (scRNA-seq) is a powerful technique that allows researchers to characterize the gene expression profile of single cells.
From each cell, mRNA is isolated, reverse-transcribed and subjected to massively parallel sequencing \cite{stegle2015computational}.
The sequencing reads are then mapped to a reference genome, whereby the number of reads mapped to each gene can be used to quantify the expression of that gene.
Alternatively, transcript molecules can be counted directly using unique molecular identifiers (UMIs) \cite{islam2014quantitative}.
Count data can be analyzed to identify new cell subtypes and to detect highly variable or differentially expressed (DE) genes between cell subpopulations.
This type of single-cell resolution is not possible with bulk RNA sequencing of cellular populations.
However, the downside is that the counts often contain high levels of technical noise with many ``drop-outs'', i.e., zero or near-zero values.
This is due to the difficulties in sequencing low amounts of RNA per cell, which decreases the capture efficiency during library preparation.
Moreover, the capture efficiency often varies from cell to cell, such that counts cannot be directly compared between cells.

Normalization of the scRNA-seq counts is a critical step that corrects for differences in capture efficiency between cells.
Two broad classes of normalization approaches are available -- those using spike-in RNA sets, and those using the counts for cellular RNA.
In the former, the same quantity of spike-in RNA is added to each cell prior to library preparation \cite{stegle2015computational}.
Any difference in the coverage of the spike-in transcripts must be caused by differences in capture efficiency between cells.
Normalization is then performed by scaling the counts to equalize spike-in coverage between cells.
For the methods using the cellular counts, the common assumption is that most genes are not DE across the sampled cells.
Counts are then scaled so that there is, on average, no fold-difference in expression between cells for the majority of genes.
This is the underlying concept of commonly used methods such as size factor \cite{anders2010differential} and trimmed mean of M-values (TMM) normalization \cite{robinson2010scaling}.
An even simpler approach involves scaling the counts to remove differences in library sizes between cells.

The type of normalization that can be used often depends on the characteristics of the data set.
In some data sets, spike-in data may not be present -- for example, droplet-based protocols \cite{klein2015droplet,macosko2015highly} do not allow spike-ins to be easily incorporated.
This obviously precludes the use of spike-in normalization.
The methods based on cellular counts can be applied more generally but have their own deficiencies.
Normalization by library size is insufficient when DE genes are present, as composition biases can introduce spurious differences between cells \cite{robinson2010scaling}.
Size factor or TMM normalization are more robust to DE but rely on the calculation of ratios of counts between cells.
This is not straightforward in scRNA-seq data, where the high frequency of drop-out events interferes with stable normalization.
A large number of zeroes will result in nonsensical size factors or undefined M-values in the TMM method.
One could proceed by removing the offending genes during normalization for each cell, but this may introduce biases if the number of zeroes varies across cells.

Correct normalization of scRNA-seq data is essential as it determines the accuracy of any downstream quantitative analyses.
In this article, we describe a simple deconvolution approach that improves the accuracy of the non-DE (i.e., size factor or TMM) normalization methods.
Using a variety of simple simulations, we demonstrate that this approach outperforms the direct application of those normalization methods for count data with many zeroes.
We also show a similar difference in behaviour on several real data sets, where deconvolved normalization yields results that are more biologically relevant.

\bibliographystyle{unsrt}
\bibliography{references}



\end{document}
